% add --shell-escape to pdflatex arguments.
% add following key to have keyboard shortcuts
%{
%    "key": "shift+b",
%    "command": "commandId",
%    "when": "editorTextFocus"
%},
%{
%"key": "shift+B",
%"command": "editor.action.insertSnippet",
%"when": "editorLangId == latex && editorTextFocus",
%"args": {
%    "snippet": "\\textbf{${TM_SELECTED_TEXT}$0}"
%}
%}

\documentclass[14pt]{extarticle}
\usepackage[left=2cm , right = 2cm, top=2cm]{geometry}
\usepackage{helvet}
\usepackage{parskip}
\usepackage{amsmath}
\usepackage{amssymb}
\usepackage{graphicx}
\usepackage[spanish]{babel}
\usepackage[dvipsnames]{xcolor}
\usepackage{tcolorbox} % above of the svg package
\usepackage{svg} 
\usepackage{hyperref}
\usepackage{minted}
\renewcommand{\sfdefault}{lmss}  % este activa la letra lmss
\renewcommand{\familydefault}{\sfdefault} % este activa la letra lmss
\sffamily % este activa la letra lmss
%\hyperlink{page.2}{Go to page 2}
%\newpage
%text on page 2
%\begin{figure}[htbp]
%  \centering
%  \includesvg{plot.svg}
%  \caption{svg image}
%\end{figure}

%\begin{minted}{csharp}
%    // single comment
%    \end{minted}

% f(n) = \begin{cases}
%    n/2  & n \text{ is even} \\
%    3n+1 & n \text{ is odd}
%  \end{cases}

%\begin{align}
%    \frac{d}{dx} \ln x &= \lim_{h\to 0} \frac{\ln(x+h) - \ln x}{h} \\
%    &= \ln e^{1/x} &&\text{How this follows is left as an exercise.}\\
%    &= \frac{1}{x} &&\text{Using the definition of ln as inverse function}
%   \end{align}


\begin{document}

\textbf{Formulación del Problema}: Dados $n$ trabajos $\{ j_1, j_2, ..., j_n  \}$, cada trabajo consiste en $s$ operaciones $\{ O_{j_1}, O_{j_2}, ..., O_{j_s} \}$ que deben ser procesadas en $m$ computadoras en ese orden satisfaciendo que hasta que una operación no sea totalmente realizada no puede comenzar la siguiente y una vez comenzada no puede ser interrumpida. La operación $O_{j_t}$ posee un tiempo requerido $T_{j_t}$ y una computadora $M_{j_t}$ en la que debe ser realizada la operación. En este ambiente, cada máquina puede procesar a lo más una operación a la vez. El objetivo es acabar todos los trabajos en el menor tiempo posible. 

\textbf{Aplicaciones Prácticas del Problema}: Este problema es especialmente útil en la manufacturación, suponer que para preparar un producto (trabajo) son necesarios $s$ pasos (operaciones), cada uno de estos pasos posee un tiempo que requerirá en una máquina especifica, y ha de ser paso a paso el proceso (las operaciones siguen un orden en específico), se trata de completar todos los productos en el menor tiempo posible.  

\end{document}

